\chapter{Abstract}
% Introduction
Companies in any kind of industry, be it electronics, pharmaceutics, food, or any other, need to support their operations by managing the exchange of information and resources with other companies. This flow of resources and information is commonly referred to as the supply chain. In a practical sense, the flow of products and services goes from supplier to consumer, but as supply chains grow more complex, they span multiple businesses and relationships, and are interlaced in many ways. This is a reason why supply chain management (SCM), the discipline which deals with the coordination of a supply chain, is becoming increasingly more important. The paths the information takes are not always obvious, making it hard to track either information or resources.

% Problem
Managing inventory and suppliers while maintaining safety, quality and keeping to the schedule is a difficult task. Delays are common, and a company’s finances, growth and reputation are affected. This is further aggravated by the fact that information is not always accurate or available when needed. And the fact that companies value their privacy makes it so that sharing information with others is not always simple or desirable. Furthermore, supply chains have many manual processes of inserting information into the companies systems, which are slow and error-prone.

% Why is it a problem
Provenance and traceability of a supply chain are a big objective for companies, along with improving security, efficiency and effectiveness. Most of these issues are, in great part, caused by the use of inadequate processes or even of inadequate software, which, traditionally, is centralized and often difficult to extend. The software used is also prone to synchronization problems, leading to big delays in both relaying the information and resources and making payments possible. 

% Solution
One way to approach these specific problems of supply chains is to update or extend the IT with new infrastructures, namely by using blockchain technology. Blockchain is a decentralized technology that has no single point of failure and allows for the immutable storage of verifiable data all over the computers that store it. With blockchain, information is stored sequentially and any piece of information, since the genesis of the chain is accessible anytime, anywhere. As such, blockchains are the perfect means to achieve traceability of a supply chain. At the same time they are a secure and incorruptible way to store information, with a fast synchronization time. It would also be the way to close the analog gaps, turning the chain fully digital and automated, leading to the possibility of a global overview.

This project focuses on researching whether the application of  blockchain technology to a supply chain is beneficial, to which extent, and what is needed to achieve this. It should be possible to ascertain how much this technology can really impact SCM over the traditional IT solutions, or in combination with them. A practical instance involving the application of this technology to a system will also be developed, to further demonstrate its possible benefits. In the end, the system will be analyzed against a system without blockchain both quantitatively, using performance metrics, as well as qualitatively, by checking which new functionalities were added.

The improvements that it could bring are, however, not limited to the problems described. Blockchain brings many other possibilities into the table, in many other areas than just supply chain.

\textbf{Keywords:} Supply Chain, Supply Chain Management, Industry, Blockchain, Network, Decentralization, Security, Traceability, Provenance, Information Integration, Ledger, Interoperability


%\chapter{Resumo}
%Tradução PT do abstract




