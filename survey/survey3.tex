%TODO: CHANGE TABLES CAPTIONS

\subsection{Group 3 - Supply Chain Points of Focus and Problems}
%GRAFICO Inventory management
This part of the survey focuses on finding out if the issues dug out on the background and pointed out in the Problem Statement chapter are really the ones that plague supply chain management. As such, a battery of questions including the various concerns mentioned was given to the respondents.

\subsection*{1 - Speed of Delivery, synchronization and traceability issues}

The respondents were asked the following 2 questions, as well as some affirmations to classify their agreement with.

\par\textbf{Questions:}
\begin{enumerate}
  \item In your experience, how frequently does the absence of crucial information cause delays in processes such as shipping, packaging, etc.?
  \item How important is Inventory Management for the efficiency of the supply chain?
\end{enumerate}

\par\textbf{Affirmations:}
\begin{enumerate}
%GRAFICO DAS 4 PRIMEIRAS RATE THE AFFIRMATION+FREQUENCY OF DELAYS (speed of delivery, synchronization,

\item In a supply chain, if the information gap between partners is shortened, a faster product cycle (or shorter lead time) can be attained.
\item Distribution, supply, demand and inventory planning all rely heavily on the information being both accurate and up to date

\item Supply Chain Management lacks a way to quickly and seamlessly share between companies all the information generated by the flow of assets in a Supply Chain.
\end{enumerate}

The data collected from these questions and affirmations can be visualized in the graphics from figure \ref{fig:group3_graphics} and is summarized in table \ref{table:group3_graphics}.

\begin{figure}[h]
    %\makebox[2pt]{}
    \resfig{survey_group3/inventory_management_importance}{Inventory Management Importance}
    \resfig{survey_group3/absenceinfo_delays_affirmation}{Relation between the absence of information and delays}

    \resfig{survey_group3/information_gap_affirmation}{Relation between information gap and delays}
    \resfig{survey_group3/planning_accuracy_affirmation}{Relation between management planning and information accuracy}
    \resfig{survey_group3/quickinfo_sharing_affirmation}{Affirmation about the lack of a system with good integration of information}

      \caption{Role, Industry, Years of Experience and Company Size}
    \label{fig:group3_graphics}
\end{figure}


\begin{table}[h]
  \centering
  \begin{tabular}{l|c|c|c|c|c|c|}
  \cline{2-7}
                                                                                                                                                                                                                                                            & \multicolumn{1}{l|}{Mode} & \multicolumn{1}{l|}{Median} & \multicolumn{1}{l|}{Mean} & \multicolumn{1}{l|}{\begin{tabular}[c]{@{}l@{}}Standard\\  Deviation\end{tabular}} & \multicolumn{1}{l|}{Range} & \multicolumn{1}{l|}{Skewness} \\ \hline
  \multicolumn{1}{|l|}{\begin{tabular}[c]{@{}l@{}}Affirmation 1 - A reduction\\  of the information gap\\  between partners can lead \\ to a faster product cycle.\end{tabular}}                          & 5                         & 4                           & 3,92                      & 1,26                                                                               & 3                          & -1,21                         \\ \hline
  \multicolumn{1}{|l|}{\begin{tabular}[c]{@{}l@{}}Affirmation 2 - Distribution, \\ supply, demand and inventory \\ planning all rely heavily on \\ the information being both \\ accurate and up to date.\end{tabular}}                                       & 5                         & 5                           & 4,64                      & 0,49                                                                               & 1                          & -0,62                         \\ \hline
  \multicolumn{1}{|l|}{\begin{tabular}[c]{@{}l@{}}Affirmation 3 - Supply Chain \\ Management  lacks a way to \\ quickly share between\\  companies all the information \\ from the supply chain.
  \end{tabular}} & 4                         & 4                           & 3,79                      & 0,83                                                                               & 3                          & -0,56                         \\ \hline
  \end{tabular}
  \caption{My caption}
  \label{table:group3_graphics}
  \end{table}

\pagebreak

Beginning with the analysis of the 2 questions:
\begin{itemize}
  \item The first question had the objective of finding out whether Inventory Management is an important discipline for the efficiency of the supply chain, which is an important aspect, if it is to be treated as point of possible improvement. The data was ranked from 1 to 5, with 1 being "Not at all important" and 5 being "Very Important". The question scored a mean of 4.48, with the mode and median both being 5, since most results sit on that score. There is little dispersion in the results, with most of the other answers scoring a 4 and only 1 answer scoring a 3. It can be concluded that \textbf{Inventory Management is an essential discipline for supply chain management, according to the professionals.}
  \item The second question tries to relate the absence of information with delays in the processes of sending products. The scale went from 0 to 10, which corresponded from the percentages 0 to 100\%. The final results ranged from 40\% to 90\%, with most of the answers sitting between 60\% and 80\%. The results were not very spread out, with a standard deviation of only 1.29, in the scale of 1 to 10. There weren't also any significant outlier values, with the skew being classified as normal, with a value below the treshold of the standard error. \textbf{The average result for this opinion was 68\%, which is a reasonably high percentage that we can say there might really be correlation between these 2 aspects, especially given the consensus and non-existance of outlier values. However, this can never be said with total certainty, since these are only opinions and not factual data}. 
\end{itemize}
%We know that Inventory management relates to lots of practices, and these are important


A follow-up with the analysis of the opinions about the affirmations:
\begin{itemize}
  \item The first affirmation related a reduction in the information gap with a faster product cycle. In layman's terms, this can be thought of as "if there is more information available, the products will get made faster, shipped faster and received faster". Many respondents seemed to only cautiously agree with the affirmation, not commiting to a strong answer. Even though the most popular answer was 5, the mean still was below the normal agreement level, at 3.92. There is some skew and spread, due to the lower results, but these still look like valid answers. Therefore, it can be informally generalized that \textbf{while it may be true indeed that having information available will generally help products finish their cycle faster, it might not always be the case, as there is not a strong consensus about this.}
  \item The second affirmation relates all the planning disciplines in SCM with the existance of up-to-date and accurate information. The respondents all answered with either 4 or 5, a very low spread of answers, with most of them strongly agreeing (agreement average of 4.64. It can then be concluded that having accurate and up-to-date information is essential for these planning management tasks, which, of course, affect inventory management, which we already concluded to also be essential. \textbf{It is, therefore, of the utmost importance for a supply chain to be provided timely with exact information}.
  \item The last affirmation states that supply chain management might lack a way to integrate the information provenient from the flow of assets between companies easily. However, out of the 3 affirmations, this is the one with the lowest agreement of them. While the average agreement, 3.79, is close to the one from the first affirmation, the most popular choice here was 4 (Agree), followed by 3 (Neutral). The skew is almost low enough that this would represent a normal distribution for the answer. \textbf{While it seems that not every professional agrees, and some think that there are already good ways to integrate information quickly and seamlessly, the majority still thinks that SCM is lacking on this aspect.}
\end{itemize}

\subsection*{2 - Quality assurance and traceability issues}

Finally, with the objective of ascerting whether quality assurance is also an issue, the respondents were asked 2 questions. The first, similarly to previously, was asking about the agreement with an affirmation, and the second was a question. Both the affirmation and question are as follows:

\begin{enumerate}
  \item Non-compliance with certain standards is a big issue that affects the reputation of companies working with sensitive products.
  \item In your experience, do you think that compliance with quality standards would be easier to achieve if all the products were traced as well as the processes they go through?
\end{enumerate}

\begin{figure}[h]
  %\makebox[2pt]{}
  \resfig{survey_group3/noncompliance_reputation_affirmation}{Relation between standard non-compliance and company reputation}
  \resfig{survey_group3/quality_traced_affirmation}{Relation between process traceability and quality standards compliance}

    \caption{Quality assurance issues}
  \label{fig:group3_graphics2}
\end{figure}
%GRAFICOS DAS 2 PERGUNTAS DE QUALITY ASSURANCE (traceability)

Looking at the graphics, the second question has a slightly higher agreement rate than the first one. However, both questions have very high agreement results, with absolutely no disagreement answers and only 1 neutral answer on the second question. It can be concluded that \textbf{quality standards compliance might affect a company's reputation and is also somewhat dependent on the traceability of products and processes they go through}.


\subsection*{Conclusions from Group 3}



\begin{itemize}
  \item From the first set of questions, we summarily conclude that the accurate synchronization of data and the speed of delivery (metrics stated in the problem statement) are related and an improvement in the synchronization might positively affect the speed of delivery, as well as other important metrics of a product's cycle. Though there may be solutions for this, the professionals still that they might be lacking.

  \item From the second set of questions, the only conclusion was that companies have in their interest to follow the quality standards, and traceability plays a big role in this.
\end{itemize}

The questions from this group have successfuly validated most, if not all,of the supply chain issues raised in the previous chapters. The remaining questions from the survey point towards points of improvement for these issues, as well as what possible features to implement in a system could help with these issues.