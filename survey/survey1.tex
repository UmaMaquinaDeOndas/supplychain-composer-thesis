\section{Results}

The results of the survey are analyzed in this chapter~\footnote{The results are available at:\url{https://drive.google.com/file/d/1Zyg8BGj0dV6KFKPZdshgbYHYi-m8snGX/view?usp=sharing}}. Each group of questions features the graphs and metrics of the corresponding questions, and a brief informal analysis is done, based on the values.


\subsection{Question Group 1 - General Information and Participant Classification}

%4 EM 1
%GRAFICO ROLE +
%GRAFICO INDUSTRIA +
%GRAFICO EXPERIENCIA +
%GRAFICO TAMANHO DA EMPRESA
% MAKE SURE ALL THE OPTIONS IN THE GRAPHICS ARE VISIBLE
\newcommand{\resfig}[2]{
    \begin{subfigure}{.55\textwidth}
        \centering
        \includegraphics[width=.9\linewidth]{media/#1}
        \caption{#2}
    \end{subfigure}
}

\begin{figure}[ht]
    %\makebox[2pt]{}

    \resfig{survey_group1/sc_role}{Relation to Supply Chain}
    \resfig{survey_group1/sc_industry}{Industry}

    \resfig{survey_group1/sc_experience}{Years of experience}
    \resfig{survey_group1/nr_employees}{Number of employees in company}

      \caption{Questions about the respondent's Role, Industry, Years of Experience and Company Size.}
    \label{fig:group1_graphics}
\end{figure}


Data from the participants was inquired, including: role, industry, years of experience and company size (if  they worked in one at all).

The data, as can be seen in Figure~\ref{fig:group1_graphics}, can be summarized in the following statements:

\begin{itemize}
\item Most people, about \textbf{68\% of the respondents, have an active role working in areas related to supply chain}. From these, 56\% work in a company directly involved in the processes of supply, manufacturing, logistics, retail or transports and 12\% work in consulting related to these areas. The rest are people with a profile that gives them knowledge about supply chain, be it through education or other type of contact with the field.
\item \textbf{ most common area of supply chain to work in is, by far, Transport and Logistics}, with 52\% of the respondents saying they have work experience in this field. \textbf{Health and Retail} are also common areas, with 20\% and 16\% respectively.
\item 48\% of the respondents have less than 3 years of experience, with 36\% having 3 to 5 years and 8\% having 5 to 10 years. However, in total, there were \textbf{only 2 respondents (8\%) without any years of experience in the field}.
\item 62,5\% of the respondents work in medium to big companies (more than 50 workers), from which 41,7\% work in companies with more than 1000 people and 12,5\% in companies with between 200 and 1000 people.
\end{itemize}

\subsection*{Conclusions from Question Group 1}

From this data, we can ascertain the general profile of the survey respondents. These are people who, in their majority, work in big companies related to the supply chain, in the various fields, but mostly in transport, logistics and health. The respondents, though they may have had education in this area, do not show a high number of years of experience, with most of them having only up until 5 years of experience.