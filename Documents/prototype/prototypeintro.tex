

Following the conclusions and requirements elicited from the previous chapter, it is now time to answer the remaining two questions. The first question, \textbf{"What is the Blockchain tool or framework most adequate to the development of an architecture that can support these requirements?"} requires an analysis of the frameworks. The second question, \textbf{"Is it possible to build a feasible architectural design, by using such a tool, to implement all these requirements?"} requires that the elicited requirements be formed into a list and that an architectural design be built and implemented using the chosen framework. Only at the end of these tasks can we verify if all the requirements are implemented by the PoC.

This chapter deals with explaining these tasks sequentially. The first section analyzes which frameworks best satisfy the requirements and a choice is made. The second through fourth sections resemble a software engineering approach: starting in the requirements specification, following it up with the design, implementation using the framework and finally, the validation of the requirements.

%
%However, due to time constraints, and also because the results from the survey took some time to collect, the requirements were written during the collection of these results and refined until after the results were finished being gathered and analyzed. At the same time, the implementation for these requirements was also being done, as it was more practical and efficient than to leave it for the end. Because of this, some requirements ended up being more emphasized in the design and implementation than some others that the survey results would indicate to be more important.